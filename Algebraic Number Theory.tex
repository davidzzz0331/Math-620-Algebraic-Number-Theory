\documentclass{article}
\usepackage[utf8]{inputenc}
\usepackage{amsmath}
\usepackage{amsfonts}
\usepackage{amssymb}
\usepackage{tikz}
\usepackage{fullpage}
\usepackage{tikz-cd}
\usepackage{spectralsequences}
\usepackage{adjustbox}
\usepackage{xfrac}
\usepackage{tcolorbox}
\usepackage{xcolor}
\usepackage{hyperref}
\usepackage{graphicx}
\graphicspath{ {D:/Chrome Downloads./} }
\usepackage[parfill]{parskip}
\usepackage{amsthm}
\usetikzlibrary{calc}
\theoremstyle{definition}
\newtheorem{theorem}{Theorem}[section]
\theoremstyle{definition}
\newtheorem{definition}{Definition}[theorem]
\theoremstyle{definition}
\newtheorem{remark}{Remark}[theorem]
\theoremstyle{definition}
\newtheorem{proposition}{Proposition}[theorem]
\theoremstyle{definition}
\newtheorem{lemma}[theorem]{Lemma}
\theoremstyle{definition}
\newtheorem{corollary}{Corollary}[theorem]
\theoremstyle{definition}
\newtheorem{example}{Example}[theorem]
\tikzset{curve/.style={settings={#1},to path={(\tikztostart)
    .. controls ($(\tikztostart)!\pv{pos}!(\tikztotarget)!\pv{height}!270:(\tikztotarget)$)
    and ($(\tikztostart)!1-\pv{pos}!(\tikztotarget)!\pv{height}!270:(\tikztotarget)$)
    .. (\tikztotarget)\tikztonodes}},
    settings/.code={\tikzset{quiver/.cd,#1}
        \def\pv##1{\pgfkeysvalueof{/tikz/quiver/##1}}},
    quiver/.cd,pos/.initial=0.35,height/.initial=0}
\title{Math 620: Algebraic Number Theory}
\author{David Zhu}

\begin{document}
\maketitle

Started with Calabi's computation of $\zeta(1)$ and $\zeta(2)$ with an ingenious integral and change of variables. 



\section{Algebraic Numbers, Algebraic Integers}



\begin{tcolorbox}[colback=red!5!white,colframe=red!30!white]
\begin{theorem}[Liouville Theorem]
\label{Liouville}
If $x$ is a irrational number of degree $n$ over the rationals, then there exists a constant $c$ such that 
\[|x-\frac{p}{q}|>\frac{c}{q^n}\]
for all $p,q>0$.
\end{theorem}
\end{tcolorbox}


The remark is algebraic numbers are harder to estimate with rationals with small denominators. 


\begin{tcolorbox}[colback=yellow!5!white,colframe=yellow!30!white]
\begin{example}
The real number 
\[\alpha= \sum_{n=0}^{\infty}10^{-n!}\]
is transcendental.
\end{example}
\end{tcolorbox}
One can show the example is indeed transcendental because it violates the bound of Theorem \ref{Liouville}. 


\begin{tcolorbox}[colback=red!5!white,colframe=red!30!white]
\begin{theorem}[Apery, $\sim$ 1980 ]
The real number 
\[\zeta(3)=\sum_{n=1}^{\infty}n^{-3}\] is irrational. 
\end{theorem}
\end{tcolorbox}


\begin{tcolorbox}[colback=red!5!white,colframe=red!30!white]
\begin{theorem}[Thue-Siegel-Roth]
    Suppose $\alpha$ is algebraic and irrational, $\epsilon>0$. Then, there is $c(x,\alpha)$ such that 
    \[|\alpha-\frac{p}{q}|> \frac{c(x,\alpha)}{q^{2+\epsilon}}\]
with $q>0$. 
\end{theorem}
\end{tcolorbox}
Note that the proof is not effective. 

\subsection{Continued Fractions}


\begin{tcolorbox}[colback=red!5!white,colframe=red!30!white]
\begin{theorem}
    Quadratic irrationals are characterized by having infinite conitnued fractions that are evetually periodic. 
\end{theorem}
\end{tcolorbox}



\begin{tcolorbox}[colback=red!5!white,colframe=red!30!white]
\begin{theorem}[Hurwitz]
    \label{Hurwitz}
    If $\alpha $ is irrational, then there are infinitely many $\frac{p}{q}$
    \[|\alpha-\frac{p}{q}|<\frac{p}{\sqrt{5}q^2}\]
    Moreover, $\sqrt{5}$ is the best bound. 
\end{theorem}
\end{tcolorbox}


\begin{tcolorbox}[colback=green!5!white,colframe=green!30!white]
\begin{remark}
The `Lagrange Spectrum' says something about how difficult to approximate an irrational by rationals. The are related to the constant $\sqrt{5}$ appearing in Theorem \ref{Hurwitz}.
\end{remark}
\end{tcolorbox}













\begin{tcolorbox}[colback=purple!5!white,colframe=purple!75!black]
\begin{definition}
The \textbf{Markov triple} is a triple $(m,n,p)$
such that 
\[m^2+n^2+p^2=3mnp\]
A \textbf{Markov number} is any number appearing in a Markov triple. 
\end{definition}
\end{tcolorbox}
These Markov triples are related to algebraic geometry of $K3$ surfaces.  


\section{Integrality}


\begin{tcolorbox}[colback=purple!5!white,colframe=purple!75!black]
\begin{definition}
Suppose $A\leq R$ are commutative rings. An $x\in R$ is \textbf{integral} over $A$ if satisfies a monic polynomial with coefficients in $A$. 
\end{definition}
\end{tcolorbox}



\begin{tcolorbox}[colback=yellow!5!white,colframe=yellow!30!white]
\begin{example}
$R:=F[u,v]/(v^2-(u^2+au+b))$ defines an elliptic curve, and $R$ is integral over $F[u]$. 
\end{example}
\end{tcolorbox}

Note that given a ring extension $A\to B$, elements in $B$ integral over $A$ forma  subring of $B$. 


\begin{tcolorbox}[colback=purple!5!white,colframe=purple!75!black]
\begin{definition}
Given a ring extension $A\to B$, the integral closure of $A$ with respect to the extension is the subring of $B$ that contains the integral elements over $A$. 
\end{definition}
\end{tcolorbox}



\begin{tcolorbox}[colback=purple!5!white,colframe=purple!75!black]
\begin{definition}
A \textbf{number field } is a finite extension of $\mathbb{Q}$. 
\end{definition}
\end{tcolorbox}
By the primitive element theorem, we know every nymber field is of the form $\mathbb{Q}[u]$ for some primitive $u$. 


\begin{tcolorbox}[colback=yellow!5!white,colframe=yellow!30!white]
\begin{example}
A \textbf{Kummer extension} is a number field of the form 
\[\mathbb{Q}[x]\]
where $x^n-a=0$ for some $a\in \mathbb{Q}$. 

\end{example}
\end{tcolorbox}


\begin{tcolorbox}[colback=red!5!white,colframe=red!30!white]
\begin{theorem}[Kronecker-Weber]
    The abelian number fields over the rationals are subfields of the cyclotomic number fields $\mathbb{Q}(\zeta_n)$. 

\end{theorem}
\end{tcolorbox}


\begin{tcolorbox}[colback=purple!5!white,colframe=purple!75!black]
\begin{definition}
The \textbf{ring of integers} $\mathcal{O}_K$ associated to a number field $K$ is the integral closure of $\mathbb{Z}$ in $K$. Alternatively, it is the subring of all algebraic integers in $K$. 
\end{definition}
\end{tcolorbox}

















\end{document}
